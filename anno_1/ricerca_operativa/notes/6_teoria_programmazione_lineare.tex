% WARN: partito da pdf nuovo, riguardare i pdf delle lezioni precedenti
% TODO: arrivato a pagina 29
%
\chapter{Teoria della programmazione lineare}

I problemi di programmazione lineare(PL) in forma scalare canonica sono sempre problemi di massimo con vincoli di minore uguale e con vincoli non negativi.

In forma scalare i problemi di programmazione lineare sono espressi come:
\begin{align}
	 & \max \sum_{j=1}^n c_j x_j                              \\
	 & \sum_{j=1}^n a_{ij} x_j \leq b_i \quad i = 1, \dots, m \\
	 & x_j \geq 0 \quad j =1, \dots, n
\end{align}

Mediante l'uso dei \textbf{vettori} osservo che:
\begin{align}
	cx = \sum_{j=1}^n c_j x_j \\
\end{align}
\begin{align}
	a_i x = \sum_{j=1}^n a_{ij} x_j
\end{align}

In questo modo posso riscrivere il problema di programmazione lineare come:
\begin{align}
	 & \max cx                              \\
	 & a_i x \leq b_i \quad i = 1, \dots, m \\
	 & x \geq
\end{align}

Anche le matrici vengono in aiuto, considerando una matrice $A \in \mathbb{R}^{m\times n}$ che ha tante \textbf{righe} quanti i vincoli (m) e la ci $i$-esima riga è il vettore $a_i$.

Si considera anche il vettore $b=(b_1, \dots, b_m) \in \mathbb{R}^{m}$ di dimensione $m$ che contiene i termini noti dei vincoli.

Nota che $Ax = (a_1 x, \dots, a_m x)$, la reppresentazione \textbf{matriciale} è :
\begin{align}
	 & \max cx   \\
	 & Ax \leq b \\
	 & x \geq 0
\end{align}


\section{PL canonici $\equiv$ PL generici}

\subsection{Trasformazione da $\min$ a $\max$}

\begin{align}
	\min cx \equiv - \max -cx
\end{align}

\subsection{Trasformazione da $\geq$ a $\leq$}

\begin{align}
	a_i x \geq b_i \equiv -a_i x \leq -b_i
\end{align}

\subsection{Trasformazione da $=$ a $\leq$}

\begin{align}
	a_i x = b_i \equiv \begin{cases}
		                   a_i x \leq b_i \\
		                   -a_i x \leq -b_i
	                   \end{cases}
\end{align}


\subsection{Sostituzione variabili $\leq 0$ con variabili $\geq 0$}

Data una variabile $x_j \leq 0$ si sostituisce con $x_j = -x_j'$ dove $x_j' \geq 0$.


\subsection{Sostituzione variabili libere con differenza di variabili $\geq 0$}

Avendo una variabile $x_j$ libera si sostituisce con $x_j' - x_j''$ con $x_j', x_j'' \geq 0$.


\section{Insiemi}
\subsection{Insieme convesso}

Un'insieme $C \subseteq \mathbb{R}^n$ è convesso se:
\begin{align}
	\forall x_1, x_2 \in C, \quad \forall \lambda \in [0,1] : \quad \lambda x_1 + (1-\lambda) x_2 \in C
\end{align}

Il segmento che unisce i due punti fa parte dell'insieme.

\subsection{insiemi limitati e chiusi}

Un'insieme $C$ è limitato se esiste una sfera di raggio finito che contiene $C$.


Un'insieme $C$ è chiuso se contiene tutti i suoi punti di frontiera.

\subsection{Semispazio e iperpiano}

Si definisce semispazio in $\mathbb{R}^n$ l'insieme di punti che soddisfano una disequazione lineare del tipo:
\begin{align}
	\sum_{j=1}^n w_j x_j \leq v
\end{align}

In forma vettoriale posso riscrivere come:
\begin{align}
	wx \leq v
\end{align}


Si definisce iperpiano in $\mathbb{R}^n$ l'insieme di punti che soddisfano una disequazione lineare del tipo:
\begin{align}
	\sum_{j=1}^n w_j x_j = v
\end{align}

In forma vettoriale posso riscrivere come:
\begin{align}
	wx = v
\end{align}


\subsection{Poliedro e politopo}

Si definisce poliedro l'insieme di un numero finito di semispazi e/o iperpiani, se il poliedro è limitato, l'intersezione prende il nome di politopo(letto come politopò e non politòpo).



\section{Regione ammissibile}

La regione ammissibile $S_a$ di un problema di PL in forma canonica è un poliedro:
\begin{align}
	S_a = \{x \in \mathbb{R}^n : a_i x \leq b_i,  i = 1, \dots, m, x \geq 0 \}
\end{align}


Si noti che:
\begin{itemize}
	\item I sempispazi e iperspazi sono insiemi chiusi
	\item L'intersezione di insiemi chiusi è un insieme chiuso
\end{itemize}

I poliedri sono insiemi chiusi, quindi anche la regione ammissibile $S_a$ è un'insieme chiuso.


\subsection{Convessità}

Siano $x, y \in S_a$:
\begin{align}
	a_i x \leq b_i,  i = 1, \dots, m, x \geq 0 \\
	a_i y \leq b_i,  i = 1, \dots, m, y \geq 0
\end{align}


Quindi $\forall \lambda \in (0,1)$ e $\forall i = 1, \dots, m$:
\begin{align}
	a_i [\lambda x + (1-\lambda) y] & =                               \\
	                                & = \lambda b_i + (1-\lambda) b_i \\
	                                & = b_i
\end{align}



Inoltre:
\begin{align}
	\lambda x + (1-\lambda) y \in S_a
\end{align}

Che ci fa capire che $S_a$ è convesso.


\subsection{Limitatezza e illimitatezza}


La regione ammissibile Sa di un problema di PL è un poliedro e come tale è un insieme chiuso e convesso. Inoltre, può essere un insieme vuoto, un insieme limitato (politopo) oppure un insieme illimitato.


\section{Vertici della regione ammissibile}

Si definisce vertice di $S_a$ un punto $\bar{x} \in S_a$ tale che $\nexists x_1, x_2 \in S_a, x \neq x_2$ tali che:
\begin{align}
	\bar{x} = 0.5 x_1 + 0.5 x_2
\end{align}

Ovvero $\bar{x}$ è il punto medio del segmento che unisce $x_1$ e $x_2$.



Ne derica che se un problema di programmazione in forma canonica ha $S_a\neq \emptyset$, allora ha almeno un vertice.


\section{Raggi}
Nel caso $S_a$ sia un poliedro illimitato, si definisce \textbf{raggio} di $S_a$ un vettore $r$ tale che:

\begin{align}
	\forall x_0 \in S_a, \quad \forall \lambda \geq 0, \quad x_0 + \lambda r \in S_a
\end{align}

La semiretta con origine in $x_0$ e direzione $r$ è completamente contenuta in $S_a$ per qualunque valore di $x_0 \in S_a$.



\section{Raggi estremi}
Un raggio $r$ di $S_a$ è detto \textbf{estremo} se non esistono altri due raggi $r_1$ e $r_2$ con direzioni distinte tali che:
\begin{align}
	r_1 \neq \mu r_2 \forall \mu \in \mathbb{R}
\end{align}

tali che:
\begin{align}
	r = \frac{1}{2} (r_1 + r_2)
\end{align}

$S_a$ ha sempre un numero finto di raggi estremi.



\section{Teorema di $S_a$}

Sia dato un problema di Programmazione lineare in forma canonica con $S_a \neq \emptyset$.
Siano $v_1, \dots, v_k$ i vertici di $S_a$ e nel caso in cui $S_a$ sia un poliedro illimitato, siano $r_1, \dots, r_h$ i raggi estremi di $S_a$.

Allora $x \in S_a$ se e solo se:
\begin{align}
	\exists \lambda_1, \dots, \lambda_k \geq 0, \quad \sum_{i=1}^k \lambda_i = 1, \quad \exists \mu_1, \dots, \mu_h \geq 0
\end{align}

tali che:

\begin{align}
	x = \sum_{i=1}^k \lambda_i v_i + \sum_{j=1}^h \mu_j r_j
\end{align}

NOTA:
I punti in $S_a$ sono tutti e soli i punti ottenibili come somma di:
\begin{itemize}
	\item una combinazione convessa dei vertici di $S_a$
	\item una combinazione lineare con coefficienti non negativi dei raggi estremi di $S_a$
\end{itemize}
Quindi un numero finito di oggetti (vertici e raggi estremi) mi permettono di rappresentare tutto l’insieme $S_a$.



\section{Insieme delle soluzioni ottime di $S_{ott}$}

\begin{align}
	S_{ott} = \{ x^* \in S_a : cx^* \geq cx \forall x \in S_a \}
\end{align}

L'insieme delle soluzioni ottime è un sottoinsieme di $S_a$.


\subsection{Forme di $S_{ott}$}

\begin{itemize}
	\item $S_{a} = \emptyset \implies S_{ott}$
	\item $S_{a} \neq \emptyset$ e politopo(insieme chiuso e limitato) $\implies S_{ott} \neq \emptyset$:
	      \begin{itemize}
		      \item $S_{ott}$ ha un solo punto
		      \item $S_{ott}$ ha infiniti punti
	      \end{itemize}
	\item $S_{a} \neq \emptyset$ e poliedro illimitato:
	      \begin{itemize}
		      \item $S_{ott} = \emptyset$ obiettivo illimitato(sequenza infinita di punti che aumento la funzione obiettivo all'infinito)
		      \item $S_{ott}$ è costituito da un solo punto
		      \item $S_{ott}$ è costituito da un'insieme infinito e limitato di punti
		      \item $S_{ott}$ è costituito da un'insieme infinito e illimitato di punti
	      \end{itemize}
\end{itemize}
