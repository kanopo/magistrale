% \chapter{Capitolo 4} % (fold)
% \label{chap:Capitolo 4}
% % TODO: partito da pagina 75
% % TODO: finito a pagina 96
%
%
% \section{Verifica di ottimalità} % (fold)
% \label{sec:Verifica di ottimalità}
%
%
%
% % section Verifica di ottimalità (end)
%
%
% \section{Verifica di illimitatezza} % (fold)
% \label{sec:Verifica di illimitatezza}
%
% % section Verifica di illimitatezza (end)
%
% \section{Cambio di base} % (fold)
% \label{sec:Cambio di base}
%
% % section Cambio di base (end)
%
% \section{Algoritmo del simplesso} % (fold)
% \label{sec:Algoritmo del simplesso}
%
% \begin{itemize}
%   \item \textbf{Inizializzazione}: Sia $B_0$ una base ammissibile e $k = 0$
%   \item \textbf{Passo 1 - Verifica ottimalità}: Se soddisfo condizione di ottimalià mi fermo e ho soluzione ottima 
%   \item \textbf{Passo 2 - Verifica di illimitatezza}: Se è soddisfatta la condizione di illimitatezza mi fermo, l'obiettivo del problema è illimitato e $S_{ott} = \emptyset$
%   \item \textbf{Passo 3 - Scelta variabile entrante in base}: TODO: aggiungere regola(\autoref{sec:Cambio di base})
%   \item \textbf{Passo 4 - Scelta variabile uscente dalla base}: TODO: aggiungere regola(\autoref{sec:Cambio di base})
%   \item \textbf{Passo 5 - Operazione di cardine}: Generata la nuova base torno al passo 1
% \end{itemize}
%
% % section Algoritmo del simplesso (end)
%
%
%
% % chapter Il metodo del simplesso (end)
