\chapter{Telecomunicacion network basics}
\section{The OSI and Internet models}
\section{Communication models}
\section{Delimitation}
\section{Sequence control}
\section{Error management}

Il controllo dell'errore ha 3 possibili soluzioni:
\begin{itemize}
	\item \textbf{Error detection}: rilevazione dell'errore
	\item \textbf{Error correction}: correzione dell'errore
	\item \textbf{Error recovery}: recupero dell'errore
\end{itemize}

%TODO: skippato da pagine 61, DA FARE
% WARN: PARTO DA PAGINE 69
%WARN: finire da pagina 79

\subsection{Complement sum}

\begin{figure}[!ht]
	\centering
	% TODO: ingrandire immagine a 0.4
	\includegraphics[width=0.3\columnwidth]{./images/complement_sum.png}
	\caption{Complement sum}
	\label{fig:complement_sum}
\end{figure}

Quando si riceve il pacchetto, si calcola il \textbf{checksum} dei dati ricevuti(come in \autoref{fig:complement_sum})
e lo si confronta al checksum allegato al pacchetto ricevuto,
nel caso di checksum differente si deve ritrasmettere il pacchetto.

\subsubsection{Other codes}

\textbf{Polynomial codes} conosciuti anche come \textbf{Cyclic Redundancy Check(CRC)},
usano moltiplicazioni tra polinomi per effettuare il checksum.

\subsection{Error correction}
Con la \textbf{block parity check} si possono recuperare errori ma solo se presente un errore di 1 bit.

Vengono quindi introdotte tecniche \textbf{Forward Error Correction(FED)}(\href{https://en.wikipedia.org/wiki/Viterbi_algorithm}{ad esempio Algoritmo di Viterbi})
che permettono di capire la presenza di un errore mediante algoritmi di ricostruzione.

Con FED si ricorre a ridondanza per eliminare errori(pochi in numero), non sono necessari messaggi
di corretta ricezione, che torna molto utile nel caso di comunicazione unidirezionale.


\begin{figure}[!ht]
	\centering
	% TODO: ingrandire immagine a 0.4
	\includegraphics[width=0.3\columnwidth]{./images/repetition_code.png}
	\caption{Repetition code}
	\label{fig_repetition_code}
\end{figure}

\section{Error recovery}

Quando si parla di comunicazione in reti di comunicazioni, si ricade nel richiedere automaticamente
un pacchetto che non risulta corretto al ricevitori,
esistono approcci automatici come \textbf{Automatic Repeat Request, ARQ}.

Esistono inoltre differenti meccanismi di ritrasmissione che come punto focale hanno:
\begin{itemize}
	\item Error detection
	\item Acknowledgements
	\item timers
	\item IU identifiers
\end{itemize}


Le procedure ARQ cambiano in base alla dimensione delle finestra:
\begin{itemize}
	\item \textbf{Stop and wait}: finestra di dimensione 1, si attende l'ack prima di inviare il pacchetto successivo
	\item \textbf{Sliding window, go-back-N}: finestra di dimensione N, si inviano N pacchetti prima di attendere l'ack(non ha un selettore per il resending e invia tutto il blocco)
	\item \textbf{Sliding window, selective repeat}: finestra di dimensione N, si inviano N pacchetti prima di attendere l'ack(ha un selettore per il resending e invia solo il pacchetto corrotto)
\end{itemize}


\subsubsection{Stop and Wait}

Il pacchetto \textbf{ACK(acknowledgement)} solitamente è molto corto per evitare
correzzioni nel pacchetto che conferma la corretta ricezione.

È necessario stabilire un tempo limite entro il quale si da per scontato
la \textit{scomparsa} del pacchetto, solitamente si basa sul \textbf{Round Trip Time(RTT)} che
dipende dalla congestione  della rete e ne misura i ritardi per arrivare da punto A a punto B.

Altro fattore chiave è capire quali dati sono stati inviati e quali no, per evitare
duplicazioni.
Per questo problema di è scelto di indicizzare i pacchetti con una sequenza che prende
il nome di \textbf{SeQuence Number(SQN)} per identificare univocamente quali pacchetti da ritrasmettere.

Si può parlare anche di ACK comulativi mediante l'uso di SQN consecutivi.

\begin{figure}[!ht]
	\centering
	% TODO: ingrandire immagine a 0.4
	\includegraphics[width=0.3\columnwidth]{./images/esempio_comunicazione_stop_wait_no_sqn.png}
	\caption{Esempio comunicazione stop and wait senza SQN}
	\label{fig:esempio_comunicazione_stop_wait_no_sqn}
\end{figure}


\begin{figure}[!ht]
	\centering
	% TODO: ingrandire immagine a 0.4
	\includegraphics[width=0.3\columnwidth]{./images/esempio_comunicazione_stop_wait_sqn.png}
	\caption{Esempio comunicazione stop and wait con SQN}
	\label{fig:esempio_comunicazione_stop_wait_sqn}
\end{figure}



\subsubsection{Stop and wait performance}

I tempi considerati sono:
\begin{itemize}
	\item \textbf{$T_U$}: tempo di trasmissione di un pacchetto, misurato in $s/IU$
	\item \textbf{$T_P$}: tempo di propagazione di un pacchetto, misurato in $s/IU$
	\item \textbf{$T_{A}$}: tempo di trasmissione di un ACK, misurato in $s/IU$
\end{itemize}

Il tempo totale per inviare un'unità informativa(caso ideale):
\begin{align}
	T_{tot} = T_{U} + 2T_{P} + T_{A}
\end{align}

Il massimo grado di utilizzo di un canale di comunicazione nel caso di \textbf{assenza di errore}:
\begin{align}
	\rho_0
	 & =\frac{T_U}{T_{tot}}                                         \\
	 & = \frac{T_U}{T_U + 2T_P + T_A}                               \\
	 & = \begin{cases}
     \frac{1}{2 + 2 \frac{T_P}{T_U}} \quad \text{se}\quad T_U = T_A  \\
		     \frac{1}{2 \frac{T_P}{T_U} + 1} \quad \text{se}\quad T_U >> T_A \\
		     0 \quad \text{se}\quad T_P >> T_U
	     \end{cases}
\end{align}



Nel caso di \textbf{presenza di errore}, non viene ricevuto l'ACK dal trasmettitore, devo fare alcune assunzioni:
\begin{itemize}
  \item Indipendenza statisticamente dei pacchetti informativi
  \item perdita di pacchetti ACK
\end{itemize}

Indico con $p$ la probabilità di perdita del pacchetto.
%WARN: finire da pagina 79


