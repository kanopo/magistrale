% TODO: ARRIVATO A PAGINA 20
% PARTITO DA PAG 28

% TODO: GUARDARE DA PAGINA 46 A PAGINA 53 per scarsa attenzione

%  TODO: arrivato a pagina 82
% TODO: prof arrivato almeno a pagina 85


\chapter{Intra-vehicle Communications}
% Bus systems: basics
%   Protocols
%   K-Line
%   CAN
%   LIN
%   FlexRay
%   MOST
%   In-car Ethernet
% ECUs
% Safety

\section{Bus Systems}

% \subsection{ISO/OSU Layers: Router}
%
% \begin{figure}[!ht]
%   \centering
%   \includegraphics[width=0.4\textwidth]{./images/osi_router.png}
%   \caption{OSI Router}
%   \label{fig:osi_router}
% \end{figure}
%
% Da notare che se il router ha la parte di network separate, allora i due lati edllo strato network hanno protocolli diversi e il router deve fare da nterprete.
%
%
% \subsection{ISO/OSI Layers: Functions in Detail}
%
% \begin{itemize}
%   \item \textbf{Layer fisico}: trasmissione dei bit
%   \item \textbf{Layer data link}: trasmissione dei frame
%   \item \textbf{Layer network}: trasmette dei pacchetti
%   \item \textbf{Layer transport}: trasmissione sicura dei frammenti
%   \item \textbf{Layer session}: gestione della sessione
%   \item \textbf{Layer presentation}: definisce sintassi e semantica dell'informazione
%   \item \textbf{Layer application}: comunicazione fra applicazioni
% \end{itemize}



\subsection{Perchè usare i bus?}
Un bus che collega tutti i componenti al posto di avere una topologia a grafo completo ha i seguenti vantaggi:
\begin{itemize}
	\item \textbf{Riduzione dei costi}: meno cavi e meno connettori
	\item \textbf{Riduzione del peso}: meno cavi
	\item \textbf{Riduzione del volume}: meno cavi
	\item \textbf{Alta modularità}: modifica veicoli
	\item \textbf{Alta modularità}: cooperazione con OEM
	\item \textbf{Modularità}: riuso di moduli
	\item \textbf{Standardizzazione}: standardizzazione dei componenti e dei protocolli (meno errori scemi)
\end{itemize}




\subsection{Casi d'uso per intra-vehicle communications}

\begin{itemize}
	\item Driveline: Engine and transmission control
	\item Active Safety: Electronic Stability Programme (ESP)
	\item Passive Safety: Air bag, belt tensioners
	\item Comfort: Interior lighting, A/C automation
	\item Multimedia and Telematics: Navigation system, CD changer
\end{itemize}

La geolacaliuazione è fornita da protocolli di navigazione satellitare. I più famosi sono:
\begin{itemize}
	\item GPS: USA
	\item Galileo: EU
	\item Glonass: Russia
	\item Beidou: China
\end{itemize}

Altre soluzioni sono RTK(Real Time Kinematic)!!



\subsection{Classificazione: On-board-communcation}
\begin{itemize}
	\item Complex control and monitoring tasks: trasmissione dei dati tra ECUs(Engine Control Unit) e MMI(Man Machine Interface simile a HMI che sta per human machine interface)
	\item Simplification of wiring: rimpiazzare i fili di rame con bus per ridurre la complessità dei cablaggi
	\item Multimedia bus systems: Trasmette un sacco di dati per i sistemi di intrattenimento
\end{itemize}


\subsection{Classificazione: Off-board-communication(OBD connector)}
\begin{itemize}
	\item Diagnostics: diagnosi del veicolo
	\item Flashing: aggiornamento del software
	\item Debugging: debug del software
\end{itemize}

\subsection{Classificazione per casi d'uso e importanza}

\begin{table}[!ht]
	\begin{adjustbox}{width=\columnwidth,center}
		\begin{tabular}{|c|c|c|c|c|c|c|}
			\hline
			Application            & Message Length & Message rate & Data rate & Latency & Robustness & Cost \\
			\hline
			Control and monitoring &                & 2            & 2         & 3       & 3          & 2    \\
			\hline
			Simplified wiring      &                &              &           & 1       & 2          & 1    \\
			\hline
			Multimedia             & 1              & 2            & 3         & 1       & 1          & 3    \\
			\hline
			Diagnosis              &                &              &           &         &            & 1    \\
			\hline
			Flashing               & 2              &              & 2         &         & 1          &      \\
			\hline
			Debugging              &                & 1            & 1         & 2       &            &      \\
			\hline
		\end{tabular}
	\end{adjustbox}
	\caption{Classificazione per casi d'uso e importanza}
	\label{tab:classification_use_case}
\end{table}


\subsection{Classificazione SAE(Society of Automotive Engineers)}

\begin{table}
	\begin{adjustbox}{width=\columnwidth, center}
		\begin{tabular}{|c|c|c|c|}
			\hline
			Class & Data rate  & vantaggio                    & Dispositivi     \\
			\hline
			A     & $10kBit/s$ & Economico                    & Diagnosi        \\
			B     & $64kBit/s$ & Correzione errori            & Networking ECUs \\
			C     & $1MBit/s$  & Comunicazione in tempo reale & Drive train     \\
			D     & $10MBit/s$ & Bassa latenza                & X-By-Wire       \\
			\hline
		\end{tabular}
	\end{adjustbox}
	\caption{Classificazione SAE}
	\label{tab:classification_sae}
\end{table}



\subsection{Network Topologies}

\begin{itemize}
	\item \textbf{Repeter}: amplificazione del segnale a livello fisico
	\item \textbf{Bridge}: medium/timing adaptation, unfiltered forwarding a livello data link
	\item \textbf{Router}: medium/timing adaptation, filtered forwarding a livello network
	\item \textbf{Gateway}: medium/timing adaptation, filtered forwarding, protocol translation a livello application
\end{itemize}



\section{Bit coding}

Esistono due tipologie di encoding dell'informazione:
\begin{itemize}
  \item Non Return to Zero (NRZ)
  \item Manchester
\end{itemize}

Nella tipologia \textbf{NRZ} il valore logico $0$ è caratterizzato da un segnale basso, mentre il segnale logico $1$ è identificato da un segnale alto.

Nell'encoding di tipo \textbf{Manchester} si pone attenzione al cambio di livello per attribuire il valore logico, il valore logico $0$ è identificato dal passaggio da basso livello ad alto livello e il valore logico $1$ è identificato dal passaggio di stato da alto livello a basso livello.



\subsection{Reducing ElectroMagnetic Interference(EMI)}

\begin{itemize}
  \item Aggiungere schermatura ai fili
  \item usare fili twistati per coppie di fili(annullano effetti di elettromagnetismo a vicenda)
  \item Ridurre la ripidità del segnale
  \item usare usare NRZ che ha pochi cambi di stato
\end{itemize}

\subsection{Clock drift}

Il clock drift è causato dalla costruzione fisica del quarzo usato per il clock, che diffferensce leggermente da altri clock, questo fenomeno causa \textbf{desincronizzazione}.
\subsection{Bit stuffing}
Il problema associato all'utilizzo della codifica NRZ è che inviando una serie di bit costanti, in presenza di piccoli ritardi, i dati vengono ricevuti in maniera sbagliata.

Una soluzione proposta è quella del \textbf{Bit Stuffing} ed inserisce un bit extra dopo $n$ bit consecutivi.
\begin{itemize}
  \item Se ci sono 3 uni di fila, agiunge uno zero
  \item se ci sono 3 zeri di fila, aggiunge un uno
\end{itemize}


\section{Classification according to bus access}

\begin{figure}[!ht]
  \centering
  \includegraphics[width=0.4\textwidth]{./images/classification_bus_access.png}
  \caption{Classification according to bus access}
  \label{fig:classification_bus_access}
\end{figure}

\subsection{Deterministic}
\subsubsection{Centralized}
Accesso al bas di tipo \textbf{master-slave}(similare ad un sistema pooling)
\subsubsection{Decentralized}

Protocolli basati su \textbf{token}, tutti collegati a cerchio e si invia il messaggio con un token (che è tipo il bastone della parola) al ricevitore, il ricevitore manda il suo messaggio dopo nella catena insieme altoken e cosi via.

Solo il nodo con il token poò inviare il suo pacchetto di informazioni.


L'altro approccio è il \textbf{TDMA}(Time-division multiple access), si identificano i client attaccati al mezzo di comunicazione e si divide la comunicazioni a slot temporali e si può capire chi invia guardando il riferimento al clock

Grande problema di TDMA è la sincronizzazione.


\subsection{Random}
\subsubsection{Non Collision Free}
\textbf{CSMA/CA(Carrier Sense Multiple Access)/(Collision Avoidance)} misura l'energia del bus e invia quando l'energia è sotto un certo \textit{livello di riferimento}.

CSMA senza CA, se sente il canale occupato, seleziona un tempo random che chiama backoff e aspetta, dopo di che riprova ad ascolare il bus.

CSMA con CA ogni nodo conta il tempo che il noda che sta comunicando finisca, i nodi posso comunicare in qualsiasi momento e quindi ogni conteggio sarà diverso, dopo che il nodo ha finito di comunicare, ogni nodo aspetta il tempo che ha contato il precedenza.

se mentre i nodi stanno aspettando il tempo contato per trasmettere uno dei nodi finisce, si salva il tempo avanzato agli altri nodi e si utilizza quello per il prossimo check di chi tocca.

Questo sistema non è \textit{giusto} e può portare pacchetti a rimanere nella coda per tempi lunghi.



% TODO: descrivere meglio CSMA/CA

\textbf{CSMA/CD(Carrier Sense Multiple Access)/(Collision Detection)} se più nodi comunicano l'energia sul bus è maggiore del solito e i nodi si rendono conto del problema.

I nodi si fermano(per risparmiare risorse) e mandano un segnale \textbf{jamming} che è una sequenza di bit di alto livello per comunicare agli altri nodi il problema e di non comuniicare per un po.

Si applicano ai nodi dei tempi di backoff mediante strategie di backoff e si riprova.

\textbf{CSMA/CR(Carrier Sense Multiple Access)/(Collision Resolution)}:
\begin{enumerate}
  \item \textbf{Arbitration phase}: si compete per avere il canale
  \item \textbf{data}: il nodo che ha vinto il canale comunica
\end{enumerate}

E si itera questo processo ogni volta che un nodo vuole comunicare.


\subsection{Typical structure of an ECU}
\begin{figure}[!ht]
  \centering
  \includegraphics[width=0.4\textwidth]{./images/ecu.png}
  \caption{Struttura ECU}
  \label{fir:struttura_ecu}
\end{figure}





\section{Protocols}


\subsection{K-Like Bus}

Si concentra su \textbf{Layer fisico} e \textbf{Layer data link} ed è un bus bidirezionale con comunicazione su di un filo.


Principalmente usato per connettere:
\begin{itemize}
  \item ECU to Tester
  \item ECU to ECU
\end{itemize}

Lo \textbf{zero logico} ha un valore di energia inferiore al $20\%$ del massimo del bus, mentre il \textbf{uno logico} è rappresentato quando il segnale supera l'$80\%$ del valore massimo.

Questo protocollo è compatibile con \textbf{UART}(Universal Asynchronous Receiver Transmitter).




% TODO: GUARDARE DA PAGINA 46 A PAGINA 53 per scarsa attenzione


\subsection{CAN - Controller Area Network}

Protocollo realizato nel 1986 da Bosh, la topologia di rete è quella del \textbf{Bus}.

Il protocollo riesce a mantenere contemporaneamente fino a 110 nodi e i segnali possibili sono 2:
\begin{itemize}
  \item \textbf{LOW}: segnale dominante
  \item \textbf{HIGH}: segnale recessivo
\end{itemize}


CAN ha una lunghezza massima del BUS di massimo $500m$ ad una velocità di $125kBit/s$.


Successivamente nello standard \textbf{ISO 11898} si è descritto due velocità:
\begin{itemize}
  \item LOW speed CAN: fino a $125kBit/s$
  \item High speed CAN: fino a $1MBit/s$
\end{itemize}

Il protocollo CAN interessa solo i layer 1 e 2 dell'ISO/OSI stack e le sue caratteristiche principali sono:
\begin{itemize}
  \item random access
  \item collision free
  \item message oriented
  \item no indirizzi(solo broadcast o multicast)
\end{itemize}


\subsubsection{Layer fisico}
Il protocollo CAN può usare due configurazioni:
\begin{itemize}
  \item \textbf{HIGH SPEED CAN}:
    \begin{itemize}
      \item fino a $500kBit/s$
      \item 2 cavi arrotolati per ridurre interferenze da campi elettromagnetici
      \item cavi di collegamento ai nodi massimo di $30m$ (branch)
      \item Segnale tra $0$ e $2V$
      \item resistore terminale di $120\Omega $
      \item l'errore deve essere scoperto entro il tempo di 1 BIT, quindi la lunghezza è vincolata a \autoref{eq:calcolo_lunghezza_massima_filo}
    \end{itemize}
  \item \textbf{LOW SPEED CAN}:
    \begin{itemize}
      \item fino a $125kBit/s$
      \item 2 cavi per ridurre le interferenze elettromagnetiche
      \item nessuna restrizione sui cavi di branch che collegano i nodi al bus
      \item segnale tra $0$ e $5V$
    \end{itemize}
  \item \textbf{SINGLE WIRE CAN}:
    \begin{itemize}
      \item velocità fino a $83kBit/s$
      \item 1 solo cavo e massa come riferimento
      \item segnale tra $0$ e $5V$
    \end{itemize}
\end{itemize}



Avendo il la velocità alla quale si vuole trasmettere i dati, si possono ricavare i metri massimi del cavo, ad esempio avendo una velocità di $R = 500kBit/s$ i metri massimi del filo sono calcolabili come:
\begin{align}
  \frac{1}{R} \geq 2 l \cdot 10^{-8} \\
  l \leq \frac{1}{2R \cdot 10^{-8}}
  \label{eq:calcolo_lunghezza_massima_filo}
\end{align}

\subsubsection{CAN in Vehicular Networks}
\textbf{Comunicazione senza indirizzo}
\begin{itemize}
    \item I messaggi portano un identificativo del messaggio di 11 bit (CAN 2.0A) o 29 bit (CAN 2.0B).
    \item Le stazioni non hanno un indirizzo, e i frame non ne contengono uno.
    \item Le stazioni utilizzano l'identificativo del messaggio per decidere se un messaggio è destinato a loro.
    \item L'accesso al mezzo avviene utilizzando CSMA/CR con arbitrato bit per bit.
    \item Il livello di collegamento utilizza 4 formati di frame: Dati, Richiesta (Remote), Errore, Sovraccarico (controllo di flusso).
    \item Il formato dei dati nel protocollo CAN segue lo schema \autoref{fig:can_data_format}
\end{itemize}

\begin{figure}[!ht]
  \centering
  \includegraphics[width=0.4\textwidth]{./images/can_data_format.png}
  \caption{Formato dei dati CAN}
  \label{fig:can_data_format}
\end{figure}

\textbf{CSMA/CR con arbitrato bit per bit}
\begin{itemize}
    \item Evita collisioni tramite l'accesso al bus controllato dalla priorità.
    \item Ogni messaggio contiene un identificativo corrispondente alla sua priorità.
    \item L'identificativo codifica "0" dominante e "1" recessivo: la trasmissione simultanea di "0" e "1" produce un "0".
    \item Riempimento dei bit: dopo 5 bit identici, viene inserito un bit di riempimento invertito (ignorato dal ricevitore).
    \item Quando nessuna stazione sta trasmettendo, il bus legge "1" (stato recessivo).
    \item La sincronizzazione avviene a livello di bit, rilevando il bit di inizio della stazione trasmittente.
    \item Attendere la fine della trasmissione corrente.
    \item Attendere 6 bit recessivi consecutivi.
    \item Inviare l'identificativo (mentre si ascolta il bus).
    \item Osservare una discrepanza tra il livello di segnale trasmesso e rilevato.
    \item Questo indica che si è verificata una collisione con un messaggio di priorità superiore.
    \item Ritirarsi dall'accesso al bus e riprovare in seguito.
    \item Realizzazione di uno schema di priorità non pre-emptive.
    \item Garanzie in tempo reale per i messaggi con priorità più alta, ad esempio, messaggi con il più lungo prefisso "0".
\end{itemize}



Esempio di arbitrato bit per bit(\autoref{fig:can_bitwise_example}):
\begin{figure}[!ht]
  \centering
  \includegraphics[width=0.4\textwidth]{./images/can_bitwise_example.png}
  \caption{Esempio di arbitrato bit per bit}
  \label{fig:can_bitwise_example}
\end{figure}

In questo caso il client 2 si rende conto che c'è un problema e fa \textbf{back off} lasciando libero il bus.
\begin{figure}[!ht]
  \centering
  \includegraphics[width=0.4\textwidth]{./images/can_bitwise_example_pt2.png}
  \caption{Esempio di arbitrato bit per bit}
  \label{fig:can_bitwise_example_pt2}
\end{figure}

In \autoref{fig:can_bitwise_example_pt2} il client 1 si rende conto del bus impegnato da una comunicazione con priorità maggiore e fa \textbf{backoff} che porta a mantenere sul bus la comunicazione con priorità maggiore(client 3) e poi segue la trasmissione del dato che il client 3 deve trasmettere.




\subsubsection{TTCAN(Time triggered CAN)}
un problema del CAN è quello della temporizzazione e del mantenimento dei clock fra i client.

Per risolvere questo problema è stato creato il TTCAN, che ha un nodo dedicato che prende il nome di \textbf{time master} è che periodicamente manda un segnale \textbf{basic cycles} a tutti i nodi.

Con \textbf{basic cycle} si intende un numero definito di slot occupabili che vengono popolati da una fase di organizzazione per priorità all'inizio della stessa.


CAN non si può usare per real time communications.

\subsubsection{Message Filtering}

I messaggi vengono accettati in base al loro identificativo mediante l'uso di due registri:
\begin{figure}[!ht]
  \centering
  \includegraphics[width=0.4\textwidth]{./images/can_filte.png}
  \caption{Registri di filtraggio CAN}
  \label{fig:can_filter}
\end{figure}


\subsubsection{Data Format}

Il formato di dati segue:
\begin{itemize}
  \item NRZ
  \item bit stuffing
  \item il frame inizia con un bit significativo(0)
  \item il message idenfier è 11 Bit(CAN 2.0A) oppure adesso 29Bit (CAN 2.0B)
  \item Bit di controllo idenficano il tipo di messaggio e la lunghezza
  \item payload: massimo 8 Byte, trasmesso a $500kBit/s$
  \item i dati interessanti del frame sono pochi e quindi il data rate effettivo è $30kBit/s$
\end{itemize}

\begin{figure}[!ht]
  \centering
  \includegraphics[width=0.4\textwidth]{./images/can_data_format_2.png}
  \caption{Data format}
  \label{fig:can_format_2}
\end{figure}



\subsubsection{Error detection LOW LEVEL}
\begin{itemize}
    \item Il mittente verifica la presenza di livelli di segnale inattesi sul bus.
    \item Tutti i nodi monitorano i messaggi sul bus.
    \item Tutti i nodi verificano la conformità del protocollo dei messaggi.
    \item Tutti i nodi controllano il riempimento dei bit.
    \item Il ricevitore verifica il CRC.
    \item Se uno qualsiasi dei nodi rileva un errore, trasmette un segnale di errore.
      (6 bit dominanti senza bit stuffing)
    \item Tutti i nodi rilevano il segnale di errore e scartano il messaggio.
\end{itemize}





\subsubsection{Error detection HIGH LEVEL}


\begin{itemize}
    \item Il mittente verifica la ricezione di un riconoscimento.
      (Il ricevitore trasmette un bit dominante "0" durante il campo di ACK del messaggio ricevuto)
    \item Ripetizione automatica delle trasmissioni fallite.
    % \item Se il controller si rende conto di causare troppi errori:
    %     \begin{itemize}
    %         \item Sospendere temporaneamente l'accesso al bus.
    %     \end{itemize}
    \item Probabilità residua di fallimento circa $10^{-11}$.
\end{itemize}




\subsubsection{Transoport Layer}
Utile alla gestione del flusso, e gestioen dei frammenti di un singolo messaggio divisi in più data frames.

I due protocolli sono:
\begin{itemize}
  \item ISO-TP
  \item TP 2.0
\end{itemize}

\subsubsection{ISO-TP}
\begin{figure}[!ht]
  \centering
  \includegraphics[width=0.4\textwidth]{./images/can_iso_tp.png}
  \caption{ISO-TP}
  \label{fig:can_iso_tp}
\end{figure}


L'\textbf{HEADER} introduce un byte opzionale per l'adressing e da 1 a 3 byte \textbf{PCI}(Protocol Control Information) che identificano il tipo di messaggio e byte specifici al messaggio.


\textbf{Single-frame} è identificato da 1 byte PCI, l'high-niggle(4bit) è 0, e il low-nibble identifica i bytes in payload, non è possibile fare controllo di flusso.

\textbf{First-frame}: identificato da 2 bytes PCI, high-nibble è 1 e low-nibble con aggiunta di 1 byte definisce la dimensione del payload.

Dopo il primo frame il sender aspetta il \textbf{flow control frame}.

\textbf{Consecutive-frame}: 1 byte PCI, high nibble è 2, low nibble è un numero d isequenza \textbf{SN} che parte da 1

\textbf{Flow control-frame}: 3 bytes pci, high nibble è 3 e low nibble specifica lo stato del flusso \textbf{FS}
\begin{itemize}
  \item FS 1: cear send
  \item FS 2: wait
\end{itemize}

Il byte 2 specifica la block size BS e il byte 3 indica ST(separation time).




\subsubsection{TP 2.0}
Protocollo simile a TCP:
\begin{itemize}
  \item connection oriented
  \item comunicazione basata su canali
  \item specifica fasi di: setup, configurazione, trasmissione e chisura
  \item ogni ECU ha un'indirizzo
\end{itemize}




\textbf{Broadcast}
\begin{itemize}
  \item ripetizione per 5 volte per evitare perdita
  \item Byte 0: adress of desctination ECU
  \item Byte 1: operation code(broadcast response or request)
  \item Byte 2, 3, 4: Service ID(SID) 
  \item Byte 5, 6: Response (si alterna 0x5555 e 0xAAAA per dire che non si vuole aspettare risposta)
\end{itemize}



\textbf{Channel setup}
\begin{itemize}
  \item Byte 0: adress destination ECU
  \item Byte 1: Operation code(Channel request, positive response e negative response)
  \item Byte 2, 3: RX ID(presente se validity nibble di Byte 3 è 0 altrimenti non settato)
  \item Byte 4, 5: TX ID(presente se validity nibble di Byte 5 è 0 altrimenti non settato)
  \item Byte 6: application type
\end{itemize}

%  TODO: arrivato a pagina 82
% TODO: prof arrivato almeno a pagina 85
